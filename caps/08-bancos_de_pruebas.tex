\section{Bancos de pruebas}
Un \hyperlink{test_bench}{\emph{banco de prueba}} especifica una secuencia de entradas que serán aplicadas por el simulador al modelo HDL que será probado. Dicho modelo puede ser un módulo Verilog o un diseño más grande. La entidad sujeta a las pruebas se denomina \hyperlink{unit_under_test}{\emph{unidad bajo prueba (UBP)}}.

En bancos de pruebas es frecuente emplear otro tipo de sentencia concurrente, el bloque \verb|initial|. Esta sentencia se ejecuta una sola vez, en tiempo de simulación cero. Al igual que el bloque \verb|always|, se puede incluir un bloque \verb|begin-end| para declarar el código procedural.

\begin{mycode}[style=verilogstyle, caption={Banco de prueba para un circuito detector de numeros primos.}, label=lst:programa5-10]
`timescale 1 ns / 100 ps

module Vrprime_tb1();
reg [3:0] Num;
wire Prime;

Vrprimed UUT (.N(Num), .F(Prime));

    initial begin: TB
        integer i;
        for (i = 0; i <= 15; i = i+1) begin #10 
            Num = i;
        end
endmodule
\end{mycode}

Los simuladores suelen generar un fichero que especifica las formas de onda que cada señal toma durante la simulación. Dicho fichero puede ser leído por aplicaciones específicas para su visualización. El correcto funcionamiento la UBP quedará a cargo del usuario, cuya función es interpretar si las formas de onda tienen sentido. Debido a que este proceso es tedioso, es conveniente formatear la salida del simulador para que sea más amigable. Para ello se pueden usar tareas del sistema como \verb|$write| o \verb|$display| para mostrar información que se considere útil por consola o, incluso, redirigirla a un fichero.

Hay situaciones en las que la combinación de entradas que una UBP puede recibir es tan elevada. En estos casos, puede no resultar factible para el diseñador examinar los resultados obtenidos para cada combinación de entradas. En estos casos, conviene escribir un \hyperlink{self-checking_test_beng}{banco de prueba autoevaluado}