\section{Encaminamiento determinista y adaptativo}\label{sec:p03intro}\pagenumbering{arabic}

\subsection{Objetivo}\label{ssec:p03objetivo}

\subsection{Desarrollo}\label{ssec:p03desarrollo}


\begin{itemize}
    \item [\textbf{Ejercicio 1.}] \textbf{Extraer la topología de red a un fichero ”topologia.topo”, y dibujarla utilizando “InfiniBand-Graphviz”:}

    Para hacer este ejercicio, lo primero que hay que hacer es conectarse a CELLIA usando \verb|slurm|

    \item [\textbf{Ejercicio 2.}] \textbf{¿Qué topología de las vistas en clase hay construida en CELLIA? Anotar sus propiedades (nº de nodos, nº de switches, grado del switch, etc.)}

    \item[\textbf{Ejercicio 3.}] \textbf{Modificar el fichero “device\_list.c” para que
ofrezca más información acerca del HCA de un nodo.} 

    \item[\textbf{Ejercicio 4.}] \textbf{Modificar un programa con verbs para añadir
información acerca del uso de la red.} 
    
\end{itemize}