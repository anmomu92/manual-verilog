\section{Circuitos combinacionales}

\subsection{Decodificadores}
A pesar de que los decodificadores en Verilog suelen formar parte de diseños mayores, también se pueden definir y probar de manera aislada. Para diseñar un decodificador, existen las siguientes aproximaciones:

\begin{itemize}
    \item \emph{Estilo estructural.} Consistiría en definir la operación del decodificador mediante puertas lógicas. Esta aproximación daría lugar a un código difícil de leer y de mantener.
    \item \emph{Estilo de flujo de datos.} Se define el comportamiento del circuito mediante sentencias de asignación contínua. El Listing \ref{lst:programa6-2} muestra un posible diseño.

\begin{mycode}[style=verilogstyle, caption={Estilo de flujo de datos.}, label=lst:programa6-2]
module decodificador_2a4(input a0, 
                   input a1, 
                   input en, 
                   output y0, 
                   output y1, 
                   output y2, 
                   output y3);
    assign y0 = en ? ({a1,a0} == 2'b00) : 0;
    assign y1 = en ? ({a1,a0} == 2'b01) : 0;
    assign y2 = en ? ({a1,a0} == 2'b10) : 0;
    assign y3 = en ? ({a1,a0} == 2'b11) : 0;
endmodule
\end{mycode}
    
    \item \emph{Estilo conductual.} Se modela el circuito por medio de código prodedural. El Listing \ref{lst:programa6-3} muestra un posible diseño.

\begin{mycode}[style=verilogstyle, caption={Estilo conductual.}, label=lst:programa6-3]
module decodificador_2a4(input a0, 
                   input a1, 
                   input en, 
                   output reg y0, 
                   output reg y1, 
                   output reg y2, 
                   output reg y3);

    always @(*) begin
        if(en) begin
            {y3,y2,y1,y0} = 0
            case ({a1,a0})
                2'b00: y0 = 1;
                2'b00: y1 = 1;
                2'b00: y2 = 1;
                2'b00: y3 = 1; 
                default: {y3,y2,y1,y0} = 0;
            endcase
        end
    end
    
endmodule
\end{mycode}

Una vez diseñado el circuito, es hora de probarlo.

\begin{mycode}[style=verilogstyle, caption={Banco de pruebas del decodificador.}, label=lst:programa6-4, mathescape]
`timescale 1 ns / 1 ps
module tb_decodificador_2a4();

    // declaracion de tipos de datos
    reg a0_p, a1_p, en_p;
    wire y0_p, y1_p, y2_p, y3_p;
    integer i, errores;
    reg [3:0] resultado_esperado;

    decodificador_2a4 UBP (.a0(a0_p),
                     .a1(a1_p),
                     .en(en_p),
                     .y0(y0_p),
                     .y1(y1_p),
                     .y2(y2_p),
                     .y3(y3_p)
                    );
    initial begin
        errors = 0;
        for(i = 0; i < 8; i++) begin
            {en, a0, a1} = i;
            #10;
            resultado_esperado = 4'b0000;
            if(en) resultado_esperado[{a1,a0}] = 1'b1;
            if ({y3,y2,y1,y0} !== resultado_esperado) begin
                errores = errores + 1;
            end
        end
        $\$$display("Test completado con $\%$d errores", errores);
    end
    
endmodule
\end{mycode}

Si la compilación con \verb|iverilog| ha sido correcta, el banco de pruebas debería mostrar 0 errores.

\subsubsection{Tipos de decodificadores}

\begin{itemize}
    \item \emph{Decodificadores binarios.} Un decodificador binario $n$-a-$2^n$ es un circuito con una entrada de $n$ bits y una salida activa de las $2^n$ totales.
    \item \hyperlink{seven-segment_decoder}{\emph{Decodificadores de siete segmentos.}} Estos circuitos tienen una entrada de 4 bits (Decimal Codificado en Binario) y una salida de 1 bit para cada segmento del LCD.
\end{itemize}

\subsection{Codificadores binarios}
Un codificador binario es un circuito que hace la acción opuesta al decodificador binario. Si un decodificador binario convierte un código de $n$ bits en otro de $m$ bits, el codificador binario correspondiente debe ser capaz de transformar el código de $m$ bit en el de $n$ bits.

\subsection{Multiplexores}
Una operación muy común en diseño digital es escoger una fuente de datos entre varias para transferirla a su destino a través de un medio compartido, como un bus. Esta operación es tan común que tiene su propio nombre, \hyperlink{multiplexing}{\emph{multiplexación}} o \emph{mux} para abreviar.

Una definición más precisa de multiplexador es la de \textit{conmutador digital que conecta los datos de 1 entre $n$ fuentes a una sola salida}. Para ello, al multiplexor hay que conectarle una entrada que permita hacer una selección entre $n$ fuentes. La anchura de este entrada de selección tendra que ser $log_2(n)$, donde $n$ es el número de fuentes.



\subsection{Demultiplexores}
Los demultiplexores son los dispositivos digitales que realizan la operación opuesta a los multiplexores, es decir, conectar una entrada (de $b$ bits) a una de entre $n$ destinos/salidas.
\end{itemize}