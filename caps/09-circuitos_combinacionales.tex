\section{Circuitos combinacionales}

\subsection{Decodificadores}
A pesar de que los decodificadores en Verilog suelen formar parte de diseños mayores, también se pueden definir y probar de manera aislada. Para diseñar un decodificador, existen las siguientes aproximaciones:

\begin{itemize}
    \item \emph{Estilo estructural.} Consistiría en definir la operación del decodificador mediante puertas lógicas. Esta aproximación daría lugar a un código difícil de leer y de mantener.
    \item \emph{Estilo de flujo de datos.} Se define el comportamiento del circuito mediante sentencias de asignación contínua. El Listing \ref{lst:programa6-2} muestra un posible diseño.

\begin{mycode}[style=verilogstyle, caption={Estilo de flujo de datos.}, label=lst:programa6-2]
module decodificador_2a4(input a0, 
                   input a1, 
                   input en, 
                   output y0, 
                   output y1, 
                   output y2, 
                   output y3);
    assign y0 = en ? ({a1,a0} == 2'b00) : 0;
    assign y1 = en ? ({a1,a0} == 2'b01) : 0;
    assign y2 = en ? ({a1,a0} == 2'b10) : 0;
    assign y3 = en ? ({a1,a0} == 2'b11) : 0;
endmodule
\end{mycode}
    
    \item \emph{Estilo conductual.} Se modela el circuito por medio de código prodedural. El Listing \ref{lst:programa6-3} muestra un posible diseño.

\begin{mycode}[style=verilogstyle, caption={Estilo conductual.}, label=lst:programa6-3]
module decodificador_2a4(input a0, 
                   input a1, 
                   input en, 
                   output reg y0, 
                   output reg y1, 
                   output reg y2, 
                   output reg y3);

    always @(*) begin
        if(en) begin
            {y3,y2,y1,y0} = 0
            case ({a1,a0})
                2'b00: y0 = 1;
                2'b00: y1 = 1;
                2'b00: y2 = 1;
                2'b00: y3 = 1; 
                default: {y3,y2,y1,y0} = 0;
            endcase
        end
    end
    
endmodule
\end{mycode}

Una vez diseñado el circuito, es hora de probarlo.

\begin{mycode}[style=verilogstyle, caption={Banco de pruebas del decodificador.}, label=lst:programa6-4, mathescape]
`timescale 1 ns / 1 ps
module tb_decodificador_2a4();

    // declaracion de tipos de datos
    reg a0_p, a1_p, en_p;
    wire y0_p, y1_p, y2_p, y3_p;
    integer i, errores;
    reg [3:0] resultado_esperado;

    decodificador_2a4 UBP (.a0(a0_p),
                     .a1(a1_p),
                     .en(en_p),
                     .y0(y0_p),
                     .y1(y1_p),
                     .y2(y2_p),
                     .y3(y3_p)
                    );
    initial begin
        errors = 0;
        for(i = 0; i < 8; i++) begin
            {en, a0, a1} = i;
            #10;
            resultado_esperado = 4'b0000;
            if(en) resultado_esperado[{a1,a0}] = 1'b1;
            if ({y3,y2,y1,y0} !== resultado_esperado) begin
                errores = errores + 1;
            end
        end
        $\$$display("Test completado con $\%$d errores", errores);
    end
    
endmodule
\end{mycode}

Si la compilación con \verb|iverilog| ha sido correcta, el banco de pruebas debería mostrar 0 errores.

\subsubsection{Tipos de decodificadores}

\begin{itemize}
    \item \emph{Decodificadores binarios.} Un decodificador binario $n$-a-$2^n$ es un circuito con una entrada de $n$ bits y una salida activa de las $2^n$ totales.
    \item \hyperlink{seven-segment_decoder}{\emph{Decodificadores de siete segmentos.}} Estos circuitos tienen una entrada de 4 bits (Decimal Codificado en Binario) y una salida de 1 bit para cada segmento del LCD.
\end{itemize}

\subsection{Codificadores binarios}
Un codificador binario es un circuito que hace la acción opuesta al decodificador binario. Si un decodificador binario convierte un código de $n$ bits en otro de $m$ bits, el codificador binario correspondiente debe ser capaz de transformar el código de $m$ bit en el de $n$ bits.

\subsection{Multiplexores}
Una operación muy común en diseño digital es escoger una fuente de datos entre varias para transferirla a su destino a través de un medio compartido, como un bus. Esta operación es tan común que tiene su propio nombre, \hyperlink{multiplexing}{\emph{multiplexación}} o \emph{mux} para abreviar.

Una definición más precisa de multiplexador es la de \textit{conmutador digital que conecta los datos de 1 entre $n$ fuentes a una sola salida}. Para ello, al multiplexor hay que conectarle una entrada que permita hacer una selección entre $n$ fuentes. La anchura de este entrada de selección tendra que ser $log_2(n)$, donde $n$ es el número de fuentes.



\subsection{Demultiplexores}
Los demultiplexores son los dispositivos digitales que realizan la operación opuesta a los multiplexores, es decir, conectar una entrada (de $b$ bits) a una de entre $n$ destinos/salidas.
\end{itemize}

\section{Dispositivos triestado}
Los \hyperlink{threestate_device}{\emph{dispositivos triestado}} son muy importantes porque, a nivel de la placa de circuito impreso son una alternativa ampliamente utilizada a los multiplexores. Existen diferentes tipos de dispositivos triestado.

\subsection{Buffers triestado}
Es el dispositivo triestado más sencillo. También se le puede llamar \hyperlink{threestate_driver}{\emph{conductor triestado}}. El buffer triestado es como un buffer normal al que se le conecta una tercera señal llamada \hyperlink{threestate_enable}{\emph{habilitación triestado}}. Cuando esta señal está activa, el buffer actúa como un buffer corriente, conectando la entrada a la salida. Por el contrario, cuando habilitación triestado es cero, la salida pasa a un estado de alta impedancia y el buffer se comporta como si ni siquiera estuviese ahí. 

Los buffers triestado son raramente usados en chip, es decir, dentro de ASICs o FPGAs, ya que los multiplexores suelen ofrecer un mayor desempeño.

\subsection{Codificación con prioridad}
En la sección de codificadores binarios (\textcolor{red}{REFERENCIAR}) se vio que estos dispositivos seleccionan la entrada activa de entre $2^n$ para codificarla en la salida pero, ¿qué ocurre si varias entradas están activas al mismo tiempo? En este caso convendría asignar una \emph{prioridad} a cada una para que la entrada con más prioridad sea la seleccionada por el codificador. Los dispositivos que hacen esto se denominan \hyperlink{priority_encoder}{\emph{codificadores con prioridad}}.

Los codificadores con prioridad tienen una salida \hyperlink{IDLE}{\emph{OCIOSA}} que se activa si ninguna de las entradas está activa.

\subsection{Puertas OR-exclusivas y funciones de paridad}

Las puertas \hyperlink{exclusive-OR}{\emph{OR-exclusivas}} son importante en cuatro aplicaciones:

\begin{itemize}
    \item \emph{Comparación.}
    \item \emph{Generación de paridad.}
    \item \emph{Suma.}
    \item \emph{Contaje.}
\end{itemize}

\subsection{Comparadores}
La comparación de dos palabras binarias es una operación frecuentemente usada en computadores. El circuito que se encarga de comparar dos palabras binarias y de decir si son iguales o no se denomina \hyperlink{comparator}{\emph{comparador}}. Algunos comparadores son capaces de interpretar las palabras a comparar como numeros con o sin signo, y realizar comparaciones de mayor/menor o igual. En este caso hablamos de \hyperlink{magnitude_comparator}{comparadores de magnitudes}.

\subsection{Elementos combinacionales aritméticos}
Estos elementos incluyen circuitos que son capaces de llevar a cabo funciones aritmeticas, como sumas, multiplicaciones, divisiones o desplazamientos. Normalmente existen operadores que se encargan de estas operaciones, dejando al sintetizador la responsabilidad de generar el circuito a partir de dicho operador. En el caso concreto de las FPGA, estas consisten exclusivamente de TCs (LUTs) interconectadas, por lo que será responsabilidad del sintetizador adaptar el código a dicha infraestructura. Los elementos combinacionales aritméticos son los siguientes:

\begin{itemize}
    \item \hyperlink{half_adder}{\emph{Semisumador}} y \hyperlink{full_adder}{\emph{Sumador completo}}. El semisumador suma dos operandos de un bit produciendo una suma de dos bits, el bit de bajo orden es la \emph{semisuma (S)} mientras que el de alto orden es el \emph{acarreo (COUT)}. Los sumadores completos realizan la misma operación de suma pero con números de más de un bit. Un sumador completo tiene, además de los operandos y el resultado de dos bit, una señal de entrada para el acarreo de la suma del bit inmediatamente anterior \emph{CIN}.
    \item \hyperlink{ripple_adder}{\emph{Sumador con propagación.}} Estos circuitos suman palabras de $n$ bits y consisten en una cascada de $n$ sumadores completos, cada uno de los cuales se encarga de la suma de un bit. El COUT de cada sumador completo se conecta con el CIN del sumador completo siguiente (exceptuando la del último)\footnote{De ahí el nombre de este sumador, ya que el acarreo se va propagando a través del circuito.}. Además, al CIN del sumador completo menos significativo se le suele conectar un valor de cero. La desventaja de este tipo de circuitos es que son lentos, especialmente aquellos que sumen números muy grandes.
    \item \hyperlink{carry-lookahead_adder}{\emph{Sumador con anticipación de acarreo.}} Este sumador mejora el desempeño del sumador con propagación. Al igual que en el anterior, en este sumador existen una serie de fases que suman un bit cada una. Pero en este caso ocurren pueden ocurrir dos cosas.
    
    \begin{enumerate}
        \item \emph{Generación de acarreo.} Esto ocurre cuando una fase $i$ produce un acarreo de salida independientemente de que haya acarreo de entrada, es decir, cuando ambos sumandos valen uno.
        \item \emph{Propagación de acarreo.} Esto ocurre cuando se genera un acarreo siguiendo la lógica anterior, es decir, cuando ambos sumandos valen uno.
    \end{enumerate}

    \item \hyperlink{group-ripple_adders}{\emph{Sumadores con propagación grupal}}
    \item \hyperlink{group-carry_lookahead}{\emph{Sumadores con anticipación de acarreo grupal}}

    \item \hyperlink{substractor}{\emph{Restadores.}} Realizan la operación opuesta al sumador. \textcolor{red}{DESARROLLAR}.
    
    \item \hyperlink{shifter}{\emph{Desplazadores}} y \hyperlink{Rotator}{\emph{Rotadores}}. El desplazamiento y la rotación son dos operaciones muy utilizadas. El desplazamiento es la operación de mover los bits de una palabra de $n$ bits una o mas posiciones hacia la izquierda/derecha. Los bits que se caigan de dicha palabra serán reemplazados por nuevos bits (normalmente ceros). La rotación funciona de manera similar solo que en este caso los bits que se caigan retornan llenando las posiciones de los bits vacantes en el otro extremo de la palabra \footnote{Por esto, a la rotación se le suele denominar también desplazamiento circular.}. El circuito más representativo de esta categoría es el \hyperlink{barrel_shifter}{Desplazador en barril}.
    
    \item \emph{Multiplicadores}
    \item \emph{Divisores}
    
\end{itemize}