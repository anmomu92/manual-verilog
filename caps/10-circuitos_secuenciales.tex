\section{Circuitos secuenciales}

Aunque hemos comentado con profusión los circuitos combinacionales, lo que realmente impera en la práctica son los circuitos secuenciales. Mientras que hemos visto que en los circuitos combinacionales la salida depende exclusivamente de los valores de las entradas en el momento actual, en los circuitos secuenciales se tienen los valores de las entradas en el pasado. Por ejemplo, si tenemos un ventilador con diferentes velocidades y queremos aumentarla, tenemos que saber algo más aparte de una entrada que indique el incremento de la velocidad. Es decir, tendría que haber un estado que nos indique la velocidad actual del sistema para saber cómo incrementar la velocidad. Es aquí donde entran en juego las \emph{máquinas de estado.}

\subsection{Máquinas de estado.}

El funcionamiento de un sistema como el ventilador puede ser conceptualizada como un conjunto de estados interrelacionados, es decir, una \hyperlink{state-machine}{máquina de estados.}