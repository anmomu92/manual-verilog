%%%%%%%%%%%%%%%%%%%%%%%%%%%%%%%%%%%%%%%%%%%%%%%%%%%%%%%%%%%%%%%%%%%%%%%%%%%%%%%%%%%%%%%%%
%  CAPÍTULO 2
%  Referencias:
%    - https://veripool.org/ftp/verilator_doc.pdf
%    - https://zipcpu.com/tutorial/lsn-01-wires.pdf
%%%%%%%%%%%%%%%%%%%%%%%%%%%%%%%%%%%%%%%%%%%%%%%%%%%%%%%%%%%%%%%%%%%%%%%%%%%%%%%%%%%%%%%%%
\chapter{Ejemplos de uso}\label{cap:ejemplos}

%%%%%%%%%%%%%%%%%%%%%%%%%%%%%%%%%%%%%%%%%%%%%%%%%%%%%%%%%%%%%%%%%%%%%%%%
%  Sección 2.1 - Ejemplo básico
%%%%%%%%%%%%%%%%%%%%%%%%%%%%%%%%%%%%%%%%%%%%%%%%%%%%%%%%%%%%%%%%%%%%%%%%
\section{Ejemplo básico}\label{sec:basico}
Como primer ejemplo, vamos a diseñar un módulo con una interfaz muy simple consistente en una entrada y una salida. Lo que hará dicho módulo por dentro será negar la entrada de un bit y conectarla a la salida de un bit. El código, que llamaremos \verb|basico.sv|, será el mostrado en el listing \ref{lst:ejemplo-basico.sv}.

\lstinputlisting[style=verilogstyle, label=lst:ejemplo-basico.sv, caption=Inversor]{codigo/ejemplo-basico/top.sv}

% Subsección 2.1.1 - verilar
\subsection{Verilar}
El primer paso es verilar el modelo. Para ello, ejecutamos la instrucción mostrada en el Listing \ref{lst:verilar} en la raíz del proyecto.

\begin{mycode}[style=bashstyle, label=lst:verilar, caption={Instrucción para verilar el diseño.}]
verilator -Wall -cc inversor.sv
\end{mycode}

\vspace{10pt}
Como se puede ver, se han añadido algunas opciones a la instrucción:

\begin{itemize}
	\item \verb|-Wall|: Activa todos los warnings de sintaxis.
	\item \verb|-cc|: Se verila el diseño en C++ en vez de en SystemC. 
\end{itemize}

Tras esto, hay que hacer un \verb|make| para generar los ficheros objetos asociados al modelo verilado y que luego habrá que enlazar durante la fase de compilación del banco de pruebas. La instrucción es la mostrada en el Listing \ref{lst:make-objects}. Con la opción \verb|-C| indicamos a la instrucción \verb|make| que cambie de directorio durante su ejecución, mientras que con la opción \verb|-f| le indicamos el fichero que tiene que utilizar dentro de dicha ubicación.

\begin{mycode}[style=bashstyle, label=lst:make-objects, caption={Construcción de los ficheros objeto.}]
make -C obj_dir -f Vtop.mk
\end{mycode}

% Subsección 2.1.2 - envolver 
\subsection{Envolver}
El siguiente paso es escribir un banco de pruebas (\textit{testbench} en inglés) en C++ que instancie el modelo verilado y lo inyecte las señales de entrada para verificar que el diseño funciona como debe. Dicho banco de pruebas se suele llamar igual que el fichero fuente del módulo principal, pero añadiéndole el prefijo \verb|tb_|. Por lo tanto, a nuestro banco de pruebas lo llamaremos \verb|tb_inversor.cpp|, y será el que se muestra en el Listing \ref{lst:tb-ejemplo-basico}. En este código se pueden destacar una serie de puntos. El primero es la inclusión de los ficheros de cabecera que contendrán los objetos y funciones necesarios para llevar a cabo la simulación. Estos son \verb|verilated.h|y \verb|Vtop.h|. Este último es uno de los ficheros generados al verilar el modelo en el paso anterior y contiene la declaración de las variables y funciones del modelo verilado (que en C++ pasará a ser una clase), por lo que su nombre dependerá de aquel que le hayamos dado al fichero fuente. El siguiente punto importante es la línea 17, punto en el cual se inicia el contexto de la simulación. Sin entrar mucho en detalle, el contexto se encargará de contener toda la información importante relativa a la simulación, como el tiempo de simulación, entre otras cosas. Además, después se instancia de igual manera un objeto del modelo verilado. En la línea 27 comienza la lógica de la simulación. Esta consiste, en este caso, de un bucle que se ejecutará un número limitado de veces y dentro del cual estimularemos las señales de entrada del modelo verilado con diferentes valoresa lo largo de las diferentes iteraciones del bucle. Asimismo, tras estimular las señales de entrada, llevaremos a cabo una evaluación del sistema en dicho punto (véase la línea 47), es decir, el modelo actualizará los valores de las señales de salida conforme a los de entrada. Tras la evaluación, se muestran los valores de las diferentes señales en la línea 50. Por último, y tras terminar el bucle, liberamos el objeto del modelo verilado.

\lstinputlisting[style=verilogstyle, label=lst:tb-ejemplo-basico, caption={Banco de pruebas del ejemplo básico.}]{codigo/ejemplo-basico/tb_top.cpp}

% Subsección 2.1.3 - compilar 
\subsection{Compilar}
Una vez escrito el banco de prueba hay que compilarlo (véase el Listing \ref{lst:compilar-tb}). Durante la compilación es necesario incluir una serie de directorios que contienen los ficheros necesarios para que la simulación funcione. Por ejemplo, el fichero de cabecera \verb|verilated.h| que incluimos en el banco de pruebas no se encuentra en el directorio estándar de bibliotecas de C, así que hay que incluirlo mediante la opción \linebreak\verb|-I /usr/share/verilator/include/|. Lo mismo ocurre con el directorio \verb|obj_dir|. Por otro lado, el fichero correspondiente al banco de pruebas no va a ser el único fichero fuente que vamos a compilar. Por ejemplo, las funciones declaradas en el fichero de cabecera \verb|verilated.h| están definidas en el fichero \verb|verilated.cpp|, localizado en el mismo directorio que aquel, y que no está compilado por defecto. Por eso es por lo que hay que compilar este y el fichero \verb|verilated_threads.cpp| también.

\begin{mycode}[style=bashstyle, label=lst:compilar-tb, caption={Compilación del banco de pruebas.}]
g++ -I /usr/share/verilator/include/ -I obj_dir/ /usr/share/verilator/include/verilated_threads.cpp /usr/share/verilator/include/verilated.cpp tb_top.cpp obj_dir/Vtop__ALL.a -o Vtop
\end{mycode}

% Subsección 2.1.4 - simular 
\subsection{Simular}
Tras compilar se genera un fichero ejecutable que se puede ejecutar con la orden \verb|./Vtop|. Dicha ejecución crea una salida en la que se muestran los valores de las diferentes señales para cada evaluación del modelo. También podemos redirigir la salida de la simulación a un fichero para su visualización más detenida, en caso de que se quiera ejecutar la simulación durante una cantidad de pasos demasiado alta. La ejecuión del programa mostraría algo como el Listing \ref{lst:salida-ejemplo-basico}.

\lstinputlisting[style=bashstyle, label=lst:salida-ejemplo-basico, caption={Resultados de la simulación del ejemplo básico.}]{codigo/ejemplo-basico/salida.txt}

En la salida se puede observar que el diseño funciona como debe. En cada flanco de subida (\verb|clk=0|), la salida \verb|b| es la negación de la entrada \verb|a|.

% Subsección 2.1.5 - depurar 
\subsection{Depurar}
En este ejemplo sencillo se puede comprobar el correcto funcionamiento del diseño simplemente observando los valores de las señales en la salida. Por lo tanto, no es necesario generar trazas para depurar las señales. Se verá cómo hacer esto en ejemplos posteriores.

