\section{La dimensión temporal}
Verilog permite especificar un \emph{tiempo de demora} en \emph{asignaciones continuas}. Esto se hace incluyendo el carácter almohadilla (\#) y un número real después de la palabra clave \verb|assign|. El número real indica la demora en las unidades indicadas por la \emph{escala temporal}. También se puede especificar una demora en asignaciones procedurales. En este caso, la almohadilla y el número se colocan tras los símbolos \verb|<| o \verb|<=|. Para especificar la \emph{escala temporal}, se usa la directiva del compilador \verb|`timescale| con la siguiente sintaxis:

\begin{alltt}
\texttt{\`}timescale \emph{time-unit} / \emph{time-precision}
\end{alltt}

\emph{time-unit} indica la unidad que estará asociada con las demoras\footnote{e.g. si \emph{time-unit} es igual a $ns$, una demora de 2 se traducira a $2ns$.}. Por otro lado, \emph{time-precision} indica la granularidad con la que operará el simulador (sueles tener un valor de $ps$).
Estos valores serán tenidos en cuenta también por \emph{tareas y funciones del sistema} como \verb|$time|.

Otra manera de invocar una demora temporal dentro del \emph{código procedural} es usando la \hyperlink{delay_statement}{\emph{sentencia de demora}}, que es un \# seguido de un número. Es importante tener en cuenta que las demoras temporales serán tenidas en cuenta durante la simulación, pero no durante la síntesis. 